\documentclass[titlepage = firstcover]{scrartcl}
\usepackage[aux]{rerunfilecheck}
\usepackage{fontspec}
\usepackage[main=ngerman, english, french]{babel}

% mehr Pakete hier
\usepackage{expl3}
\usepackage{xparse}

%Mathematik------------------------------------------------------
\usepackage{amsmath}   % unverzichtbare Mathe-Befehle
\usepackage{amssymb}   % viele Mathe-Symbole
\usepackage{mathtools} % Erweiterungen für amsmath
\usepackage[
  math-style=ISO,    % \
  bold-style=ISO,    % |
  sans-style=italic, % | ISO-Standard folgen
  nabla=upright,     % |
  partial=upright,   % /
]{unicode-math}% "Does exactly what it says on the tin."
\usepackage[section, below]{placeins}

% Laden von OTF-Mathefonts
% Ermöglich Unicode Eingabe von Zeichen: α statt \alpha

\setmathfont{Latin Modern Math}
%\setmathfont{Tex Gyre Pagella Math} % alternativ zu Latin Modern Math
\setmathfont{XITS Math}[range={scr, bfscr}]
\setmathfont{XITS Math}[range={cal, bfcal}, StylisticSet=1]

\AtBeginDocument{ % wird bei \begin{document}
  % werden sonst wieder von unicode-math überschrieben
  \RenewDocumentCommand \Re {} {\operatorname{Re}}
  \RenewDocumentCommand \Im {} {\operatorname{Im}}
}
\usepackage{mleftright}
\setlength{\delimitershortfall}{-1sp}

%Sprache----------------------------------------------------------
\usepackage{microtype}
\usepackage{xfrac}
\usepackage[autostyle]{csquotes}    % babel
\usepackage[unicode, pdfusetitle]{hyperref}
\usepackage{bookmark}
\usepackage[shortcuts]{extdash}
%Einstellungen hier, z.B. Fonts
\usepackage{booktabs} % Tabellen


\title{Wiederholungsaufgaben}
\author{
    David Gutnikov\\
    \href{mailto:david.gutnikov@tu-dortmund.de}{david.gutnikov@tu-dortmund.de}
}

\begin{document}
    \maketitle
    \newpage    
    %\tableofcontents
    \newpage

    \section*{1. Aufgabe}
        \subsection*{1.1 Aufgabe}
            Der Mittelwert $\overline{x}$ einer normalverteilten Messgröße $x$, wie es die meisten physikalisch relevanten Messgrößen sind, ist der Wert, der am meisten erwartet wird. Also der Erwartungswert. Eine Art der Berechnung sieht wie folgt aus mit der Anzahl $N$ der Messungen $x_\text{i}$:
            \begin{equation*}
                \overline{x} = \frac{1}{N} \sum_{\text{i}=1}^{N} x_\text{i}
            \end{equation*}

        \subsection*{1.2 Aufgabe}
            Die Standardabweichung $s$ besagt inwieweit die Messwerte verteilt sind und ist wie folgt zu berechnen:
            \begin{equation*}
                s = \sqrt{\frac{\sum_{\text{i}=1}^{N} (x_\text{i} - \overline{x})^2}{N - 1}}
            \end{equation*}
        
        \subsection*{1.3 Aufgabe}
            Die Standardabweichung ist ein Maß für die Streuung der Messwerte. Der Fehler des Mittelwerts higegen unterscheidet sich von der Standardabweichung um den Faktor $\frac{1}{\sqrt{N}}$, wo $N$ wieder die Anzahl der Messwerte ist.
            \begin{equation*}
                s = \sqrt{\frac{\sum_{\text{i}=1}^{N} (x_\text{i} - \overline{x})^2}{N(N - 1)}}
            \end{equation*}

    \section*{2. Aufgabe}
        \begin{align*}
            R_\text{innen} = 10 \pm 1 \\
            R_\text{außen} = 15 \pm 1 \\
            h = 20 \pm 1           
        \end{align*}
        Das Volumen $V$ eines Hohlzylinders mit solchen Maßen wird wie folgt berechnet:
        \begin{equation*}
            V = \pi \cdot (R_\text{außen}^2 - R_\text{innen}^2) \cdot h
        \end{equation*}
        Der dazugehörige Fehler nach der Gaußschen Fehlerfortpflanzung:
        \begin{equation*}
            \Delta V = \sqrt{\biggl(\frac{\partial V}{\partial h} \Delta h \biggr)^2 + \biggl(\frac{\partial V}{\partial R_\text{innen}} \Delta R_\text{innen} \biggr)^2 + \biggl(\frac{\partial V}{\partial R_\text{außen}} \Delta R_\text{außen} \biggr)^2}
        \end{equation*}
        Also ist das Volumen ca.
        \begin{equation*}
            V = (7.9 \pm 2.3) \cdot 10^3 \, \mathrm{cm}
        \end{equation*}
        groß.

    \section*{3. Aufgabe}
        \begin{table}[h]
            \centering
            \caption{Tabelle mit Liniennummer, Spannung $U$ in $\mathrm{V}$ und Abstand}
            \label{tab:tabelle1}
            \begin{tabular}{c c c}
                \toprule
                Liniennummer & $U$ / $\mathrm{V}$ & D / $\mathrm{mm}$ \\
                \midrule
                1 & -19.5 & 0  \\
                2 & -16.1 & 6  \\
                3 & -12.4 & 12 \\
                4 & -9.6  & 18 \\
                5 & -6.2  & 24 \\
                6 & -2.4  & 30 \\
                7 & 1.2   & 36 \\
                8 & 5.1   & 42 \\
                9 & 8.3   & 48 \\
                \bottomrule
            \end{tabular}
        \end{table}
        \FloatBarrier
        \noindent
        Die Abstandswerte wurden mit der Formel
        \begin{equation*}
            D = (N_\text{Linie} - 1) \cdot 6 \, \mathrm{mm}
        \end{equation*}
        berechnet.
        \begin{figure}[h]
            \centering
            \caption{Die Werte von $D$ gegen $U$ aufgetragen und die lineare Regression zu diesen Messwerten wird hier dargestellt.}
            \label{fig:figure1}
            \includegraphics[width = 0.8\linewidth]{Regressiongraph.pdf}
        \end{figure}
        %\FloatBarrier
        \noindent
        Die Steigung $m$ und der y-Achsenabschnitt $b$ der Regressionsgeraden in Abbildung \ref{fig:figure1} haben folgende Werte:
        \begin{align*}
            m = (1.7 \pm 0.02) \, \frac{\mathrm{mm}}{\mathrm{V}} \\
            b = (33.9 \pm 0.21) \, \mathrm{mm}
        \end{align*}

\end{document}



















